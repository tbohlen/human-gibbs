\section*{Results}
\label{sec:results}

62 mechanical-turk trial results were analyzed, giving a total of 2480 moves.
The probability with which the particle filter would choose the same group that
the human did was recorded. The mean of these $2480$ log probabilities was an
unimpressive $-1379.256$, and the median was $-29.612$.

The three histograms in figures \ref{fig:fullhist}, \ref{fig:smallhist}, and
\ref{fig:cumulativehist}, though, show that $48.9%$ of the
log probabilities were greater than $-0.25$ and $51.5%$ were greater than
$-1.0$. $1216$ moves saw the particle filter inference algorithm assigning
probabilities greater than $0.5$ to the categorization selected by the human,
and the average move involved a choice between $4.929$ groups, making $20.286%$
success rate equivalent to chance. The particle filter model, therefore, agreed
with human data at least $49.032%$ of the time, performing far better than
chance.

\begin{figure}
\centering
\includegraphic[scale=1]{img/hist0.png}
\label{fig:fullhist}
\caption{A histogram showing the log probability of the particle filter making
the same move as the human for each of the 2480 human moves.}
\end{figure}

\begin{figure}
\centering
\includegraphic[scale=1]{img/hist4.png}
\label{fig:smallhist}
\caption{A histogram showing the log probability of the particle filter making
the same move as the human for each of the moves for which this value was
greater than $-500$. This is a zoomed in view, so to speak, of
\ref{fig:fullhist}.}
\end{figure}

\begin{figure}
\centering
\includegraphic[scale=1]{img/hist8.png}
\label{fig:cumulativehist}
\caption{A histogram showing the log probabilitt of the particle filter making
the same move as the human for each of the 2480 human moves, but with all moves
for which this value was less than $-9$ grouped into the lowest probability bin.}
\end{figure}

