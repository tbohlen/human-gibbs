\section{Discussion}
\label{sec:discussion}

The results of this experiment indicated promise for Baysian models, and, in particle, particle
filter models, of human cognition, which was to be expected considering the work
by Anderson, Sanborn, Griffiths, and Navarro supporting this belief. That said,
the $49.032\%$ success rate was very promising but hardly conclusive evidence
that a particle filter model could accurately model the human cognitive process.
Further analysis of the data presented here, as well as a number of refined
studies suggested below would be necessary to strengthen this claim.

The error in the particle filter selection process, when compared to human data,
could be the result of a number of errors or incorrect assumptions. The particle
filter treated each pixel of the $100$ by $100$ pixel images as a feature, presenting
two potential problems. First, this extremely high dimensionality may have been
the reason for the tendency of the filter to assign a probability of nearly $1$
on its top choice, and probability of nearly $0$ to all other choices.

More worryingly, such a treatment of the image does not fit well with a human's
tendency to extract features, even from such relatively featureless images. In
future this could be remedied by performing a feature extraction or simply
including in the likelihood model a covariance between dimensions. Future
experiments could also use more structured images which would be more easily
interpreted by the subjects, and, therefore, potentially result in better data.

The particle filter likelihood calculation involved the discretization of a
Student's $t$ distribution via discrete sampling of a continuous Student's $t$
distribution followed by a normalization of these samples. This method may not
be an acceptable mapping of the continuous function to a discrete domain.
Additionally, the lack of any quantification of the 8-group cap in the prior
probability distribution could have caused error.

Another possible source of error, which should be considered in all experiments
using web interfaces and paid participants, was the interface itself. The fact
that images layered on top of one another when added to the same group may have
skewed the human categorization behavior and perception of the groups. A future
experiment that presents half of the subjects with the average of all image in a group,
and all others with each individual image in a group and analyzed the resulting
categorization data for biases could help determine if this is a valid concern.
Other subtle interface decisions, such as the fixed area of the group boxes presented
on the web interface could have further biased the data and should be considered
carefully in future experiments. Additionally, a subset of the samples generated
from Mechanical Turk should always be checked to make sure the subjects did not
create "junk" trials simply to win their pay.

Despite these many concerns, this study did show that comparing human
categorization to a Baysian categorization model on a move by move basis is both
feasible and worthwhile.  The particle filter took approximately twelve hours to
analyze all 2480 individual moves on a 2.2Ghz Intel Core i7 processor, and
required around 100MB of memory to do so. Move-by-move analysis does indeed
offer a tougher test for Baysian categorization models, as mentioned in the
introduction, and so further use of this technique seems prudent.

One final, although quite interesting, potential direction for further
experiments involves investigating the parameters used in the model. 
A machine inference algorithm could be decisively shown to exclude some logic that a human
categorizer is using if the parameters that result in the best fit between a
human's decision and the algorithm's decision vary significantly from move to move.
Such analysis seems valuable to pursue in future experiments. Phrased
another way, significant variation in the best fit parameters indicates that the
humans had to apply additional logic to modify his internal model of the
system, and so the inference model in question is missing some human reasoning.
Such analysis seems valuable to pursue in future experiments.