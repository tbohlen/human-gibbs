\section{Introduction}
\label{sec:intro}

Rational analyses of human cognition seek to explain and quantify human behavior
and thought processes under the assumption that they are an optimal adaptation
to the constraints of the environment. Anderson \cite{anderson1991} argues that
categorization is a basic function of human cognitive processes, and that
Bayesian statistical inference is a theoretically motivated and effective model
for human categorization. Sanborn, Griffiths, and Navarro \cite{sanborn2010}
further investigate Bayesian algorithms for category learning, finding a
single-particle particle filter to be most effective, and most similar to human behavior.

To extend upon the previous work concerning particle filters and category
learning, this experiment analyzes each individual move made by a human while
sorting data, rather than simply analyzing the end result. A single end sort can
be achieved by a number of paths exponential in the size of the largest
category. Analyzing the process step by step allows access to this vast amount
of missed information, assuming it can be analyzed in some useful way. In this
way, move-by-move analysis is an exponentially tougher test for human inference
models. 